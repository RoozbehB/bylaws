\chapter{Assembly}
\section{Responsibilities of Representatives}
\begin{bylaws-number}
  \item Representatives shall serve on at least one standing or ad hoc committee of the Assembly, or a University Senate committee, for at least one full semester out of the legislative session. Committee membership shall be assigned by the Vice President for Committee Affairs as specified in Article 4.
  \item Representatives shall be aware of issues on the agenda prior to Assembly meetings, and are expected to engage in communication with their departments on such issues. Representatives are encouraged to caucus with members of their department prior to Assembly meetings.
  \item Representatives are expected to attend regularly scheduled Assembly and committee meetings. In cases where absence from an Assembly meeting cannot be avoided, Representatives are encouraged to designate proxies. In such instances, the Representative should transmit, by written or electronic communication, the name of the individual serving as his/her proxy to the Vice President for Legislative Affairs. The proxy must be a currently enrolled graduate student from the same program as the absent Representative, and the transmission of the proxy’s name must take place sufficiently in advance of the meeting for the Vice President for Legislative Affairs to verify that the individual meets these criteria. Proxy representatives shall have the same rights and privileges as the Representative they are temporarily replacing.
  \item Representatives are expected to engage in bi-directional communication with constituents, both informing them of ongoing GSG business and soliciting concerns and ideas.
  \item Representatives are expected to be familiar with the Constitution and Bylaws of the GSG, and to ensure that the provisions in these documents are observed.
  \item Representatives may be impeached according to the procedures designated in Article 3 for failure to fulfill the responsibilities of office as specified in these Bylaws. Representatives who are removed from their positions by a vote of the Assembly during an impeachment proceeding shall be barred from holding office within the GSG for at least one full legislative session.
  \item Representatives shall maintain their enrollment throughout the duration of their term. If a Representative, for any reason, finds him/herself no longer enrolled in credits toward a graduate degree, or in the case of students graduating in spring until the end of fiscal year of the year of their graduation, it is the Representative’s responsibility to resign his/her position.
  \item Any Representative who seeks to resign from his/her elected position shall give notice to the Vice President for Legislative Affairs and, when possible, seek a replacement candidate(s) from his/her program and assist with the transition of his/her position. Representatives shall communicate their intent to resign, in written or electronic form, to the Vice President for Legislative Affairs who shall include this information in his/her next monthly assembly report.
  \item All Representatives shall conduct themselves in a courteous and respectful manner. Questioning the motives or integrity of other members is strictly out of order. Repeated offenses within the Assembly shall be grounds for impeachment.
\end{bylaws-number}

\section{Attendance Standards}
\begin{bylaws-number}
  \item A Representative who cannot regularly attend Assembly or committee meetings should resign his/her position, and is encouraged to seek a replacement from his/her program.
  \item Representatives shall be aware of issues on the agenda prior to Assembly meetings, and are expected to engage in communication with their departments on such issues. Representatives are encouraged to caucus with members of their department prior to Assembly meetings.
  \item A Representative who has three consecutive, Unexcused Absences from regularly scheduled Assembly meetings, or who is absent, whether Excused or Unexcused, from more than half of the regularly scheduled Assembly meetings in a session, or fails to attend an assigned committee meeting during a semester, may be removed from his/her position.
  \item Before a Program Representative may be removed, the Vice President for Legislative Affairs must take the following steps:
  \begin{enumerate}
    \item Determine that the Representative has two consecutive, Unexcused Absences from regularly scheduled Assembly meetings;
    \item Make a reasonable effort to contact the Representative and remind him/her of attendance requirements, and explain that an additional Unexcused Absence may result in removal;
    \item Determine, following a third consecutive, Unexcused Absence, that the Representative is unresponsive.
  \end{enumerate}
  \item Once the Vice President for Legislative Affairs determines that a Program Representative is unresponsive, he/she shall remove the Representative from the roster of Program Representatives, notify the Representative of his/her removal, and inform the Elections Committee of the new vacancy. The removal shall be communicated to the Assembly in the next monthly report of the Vice President for Legislative Affairs.
\end{bylaws-number}

\section{Legislation}
\begin{bylaws-number}
  \item The Assembly shall be empowered to pass legislation.
  \item Legislation may be in the form of Resolutions, which express the opinion of the Assembly, or Acts, which concern the governing documents of the GSG or its budget.
  \item Legislation must be submitted to the Rules Committee no later than nine business days before the Assembly meeting at which the submitter wishes for the motion to be considered. An affirmative vote of a majority of the Rules Committee shall be required before any such motion is placed on the Assembly agenda.
  \item The Assembly reserves the right to discharge any pending legislation from the Rules Committee and bring it forward for immediate consideration by the entire Assembly by a two-thirds majority vote of the voting Representatives present.
  \item Any item of legislation passed by the Assembly may specify a date upon which it becomes null and void. Executive Orders passed during any given legislative session shall become null and void at the conclusion of said session.
\end{bylaws-number}

\section{Legislative Session}
\begin{bylaws-number}
  \item The legislative session for the GSG shall run from 1 September through 30 June. The Assembly shall be considered in recess during the months of July and August.
\end{bylaws-number}

\section{Non-program Representatives and others}
\begin{bylaws-number}
  \item Non-program Representatives shall be defined as representatives from graduate-majority organizations that are not graduate programs.
  \item Non-program Representatives may participate in Assembly upon receipt by VPLA of a letter from organization they represent, signed by two officers.
  \item Non-program representatives and other visiting representatives of graduate-majority organizations shall not have voting rights, but may contribute freely and openly to the Assembly after recognition by the Chair.
\end{bylaws-number}

\section{Authorization to Attend Assembly Meetings Electronically}
\begin{bylaws-number}
  \item A Representative who is physically unable to attend an Assembly meeting may request an electronic audio-only or audio-video connection to the rest of the Assembly. The Representative should transmit this request, by written or electronic communication, to the Vice President for Legislative Affairs at least one week prior to the scheduled meeting to allow for the proper technology setup.
  \item The electronic communication method used must allow the opportunity for simultaneous aural communications among all participating members equivalent to those of meetings held in one room or area. Remote participants can only be muted if they have the ability to unmute themselves.
  \begin{bylaws-number}
    \item If the presence of the member attending electronically is required for the presence of a Quorum, that member must be using bi-directional video communication in addition to audio in order to show active participation in deliberations.
  \end{bylaws-number}
  \item Electronic participation is an option that will be made available on a best-effort basis. It is not the right of an Assembly Member to electronic participation if it is not feasible to create one due to technical issues, lack of sufficient notice or any other reason.
  \item Electronic meetings are not a replacement for a central location for a meeting. At least half of the Quorum must be in the one room or area in which the meeting is held.
  \item By the last meeting of the 35th assembly, the VPLA must have two Acts ready and approved by Rules, listed in 2.5.E.1 and 2.5.E.2. Should neither act pass, this provision shall be repeated twice every year, at the second meeting of the Fall semester and the second meeting of the Spring semester.
  \begin{bylaws-number}
    \item An act amending the Bylaws to remove this provision for Authorization to Attend Assembly Electronically
    \item An act amending the Bylaws to establish rules relating to the use of electronic communications, including but not limited to, (a) methods for determining presence of a quorum; (b) the conditions under which a member may raise a point of order doubting the presence of a quorum and the conditions under which the continued presence of a quorum is presumed if no such point of order is raised; (c) methods for those participating electronically for seeking recognition and obtaining the floor; (d) means by which motions may be submitted in writing during a meeting; (e) methods for taking and verifying votes; and (f) contingencies for technical difficulties, malfunctions, or excessive electronic interruption, particularly when impacting the existence of a quorum.
  \end{bylaws-number}
\end{bylaws-number}
