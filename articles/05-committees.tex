\chapter{Standing and \textit{Ad Hoc} Committees}

\section{Standing Committees}
\begin{bylaws-number}
  \item All committees derive their authority from the Assembly. All committee decisions are subject to review within the Assembly and may be overturned by a majority vote of the Assembly, unless otherwise specified in these Bylaws.
  \item Committee membership shall be open to all graduate students, as well as administrators and staff with responsibilities pertaining to the charge of any committee, who shall be non-voting.
  \item All committees shall maintain a membership of at least three to seven members unless otherwise specified in the committee’s description below. An odd number of members is strongly encouraged for the purposes of committee voting. Committee members, if possible, shall represent diverse fields, disciplines, and areas of expertise to enhance the fair and objective operations of their committee.
  \item The Chief of Staff shall review committee memberships at the beginning of each legislative session, and reassign committee memberships in keeping with these limits, taking into account responsibilities and preferences of Representatives. A membership of at least five is strongly recommended for the Budget and Finance, Rules, and Elections Committees.
  \item All committees shall elect a chair, unless such office is already specified in the Constitution or these Bylaws.
  \item All committees shall establish standing rules. Committee standing rules shall be shared with the Chief of Staff.
  \item All committees shall meet on a regular basis.
  \item In addition to the standing committees outlined in the Constitution, the following are also standing committees of the GSG:
  \begin{bylaws-number}
    \item Legislative Action Committee \hfill \\
The Legislative Action Committee shall monitor governmental activities that may affect graduate students and seek to promote GSG positions regarding said activities. The Legislative Action Committee shall also oversee and coordinate graduate participation in the annual Terrapin Pride Day meetings with legislators in Annapolis.
    \item Social and Sport Committee \hfill \\
The Social and Sport Committee shall organize activities to further the social integration of graduate students from all programs.
    \item Academic Affairs Committee \hfill \\
The Academic Affairs Committee shall be charged with monitoring and reporting on all University policies which pertain to the academic and professional development of graduate students. The Committee shall also be responsible for reviewing and developing GSG policy and programs related to the academic and professional development of graduate students.
    \item Elections Committee \hfill \\
The Elections Committee shall be responsible for conducting all GSG elections in accordance with the rules established in these Bylaws, and for maintaining any additional procedures or policies governing elections.
    \item Diversity Committee \hfill \\
The Diversity Committee shall reach out to, build ties between, and advocate for graduate students of all races colors, creeds, sexes, sexual orientations, gender identities, marital statuses, personal appearances, ages, national origins, political affiliations, hidden or visible disabilities, and other disadvantaged groups. They shall also work to promote diversity within the GSG.
    \item Graduate Researchers, Employees, Assistants, and Teachers Committee (GREAT) \hfill \\
The Graduate Researchers, Employees, Assistants, and Teachers Committee shall investigate issues surrounding graduate employment on campus, formulate proposals on policies and procedures to address current or future problems and improve the quality of life of graduate students working on campus, and meet regularly with the Graduate Assistant Advisory Committee, University Administration and Graduate School to help implement any necessary changes. The committee shall maintain a diverse membership representing a mix of different disciplines and types of assistantships, as well as other forms of graduate employment such as lectureships. The committee may include as many members as it deems necessary to fulfill its purpose.
  \item Student Affairs Committee \hfill \\
The Graduate Student Affairs Committee shall be responsible for all issues that pertain to the quality of life of graduate students including but not limited to housing, transportation, safety, sustainability, and other relevant topics. They shall advocate on behalf of the GSG to the University to improve the quality of graduate students’ experiences.
  \end{bylaws-number}
  \item Additional Responsibilities of the Rules Committee.
  \begin{bylaws-number}
    \item The Rules Committee shall be responsible for the numbering, recording, and tracking of all legislation.
    \item The Rules Committee shall be responsible for the preparation of the Assembly agenda.
    \item The Rules Committee shall be responsible for documenting outcomes of and actions taken on resolutions from the current Assembly and the two immediate preceding Assemblies, and disseminating this information before the end of each legislative session.
    \item Procedures for Approving Multiple Guest Speakers or Presentations Exceeding 20 Minutes
    \begin{bylaws-number}
    	\item Proposals for guest speakers may come from Representatives, Executives, or members of the campus community at large. The Vice President for Legislative Affairs may reserve one guest speaker per Assembly meeting for a presentation up to 20 minutes in length. Any additional guest speakers or extended presentation lengths are subject to Rules Committee review.
    	\item The Vice President for Legislative Affairs shall notify the Rules Committee of the proposed additional guest speaker(s) and/or extended presentation(s) at the earliest possible committee meeting. The Rules Committee may recommend approval of all, part, or none of the additional guest speaker(s) and/or extended presentation(s). Majority vote is required to finalize the Rules Committee’s official recommendations.
   	\end{bylaws-number}
  \end{bylaws-number}
  \item Additional Responsibilities of the Budget and Finance Committee
  \begin{bylaws-number}
    \item The Budget and Finance Committee shall, to the best of its ability, reconcile all GSG accounts at the end of each legislative session, prior to transference of responsibility to a new chair and membership.
    \item The Chair of the Budget and Finance Committee, or designee, shall maintain proper books of accounts in both electronic and hard copy format. The chair shall transfer these books to a new chair when the office is transferred.
    \item The Budget and Finance Committee shall be responsible for approving exceptions to allow the purchase of bottled water less than one gallon, according to the procedures established in 7.3.D.
  \end{bylaws-number}
\end{bylaws-number}

\section{Ad Hoc Committees}
\begin{bylaws-number}
  \item Ad hoc committees shall be established as deemed necessary by the Assembly.
  \item Ad hoc committees shall operate under the same regulations as the standing committees, unless otherwise specified by the Assembly during their creation.
\end{bylaws-number}

\section{Committee Chairs}
The duties and responsibilities of all standing and ad hoc committee chairs shall be to:
\begin{bylaws-number}
  \item Establish, in conjunction with committee members, goals and objectives for the committees.
  \item Communicate with the Chief of Staff on a regular basis.
  \item Hold committee meetings regularly.
  \item Ensure that accurate and complete minutes are kept of all committee activities.
  \item Submit reports at each Assembly meeting that summarize committee activities to the Assembly.
  \item Orient all new committee members.
\end{bylaws-number}
