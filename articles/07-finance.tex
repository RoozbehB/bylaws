\chapter{Financial Policies And Procedures}
\label{finance}

\section{Use of the Graduate Student Activities Fee}
\begin{bylaws-number}
  \item Purpose: \hfill \\
  The Graduate Student Activities Fee (GSAF) has three distinct yet related purposes:
  \begin{bylaws-number}
    \item To support the advocacy and outreach activities of the Graduate Student Government (GSG) by:
    \begin{enumerate}[i]
      \item Funding the organization’s general operations and its administration of the GSAF allocation process.
      \item Providing compensation, in the form of graduate assistantships, hourly wages, or honoraria, to the GSG officers and staff who oversee the organization.
      \item Funding programming that creates supportive social environments and a sense of community for the entire graduate student population.
    \end{enumerate}
    \item To fund activities which contribute to the general enrichment of the graduate students’ University experience by:
    \begin{enumerate}[i]
      \item Serving the interests of the diverse student population which pays the fee.
      \item Personalizing the campus experience through the support of programs for various subpopulations.
      \item Creating opportunities for student participation in a wide range of activities.
      \item Providing opportunities for developing student leadership and civic engagement skills.
      \item Supporting the professional development and career pursuits of graduate students in their chosen fields of study.
      \item Supporting student-run learning experiences outside of the formal classroom.
    \end{enumerate}
    \item To provide graduate students with support and information regarding University policies and legal matters through the Graduate Legal Aid Office (GLAO) by:
    \begin{enumerate}[i]
      \item Funding the office’s general operations, including compensation for office staff, which may take the form of salaries, graduate assistantships, or hourly wages.
      \item Granting oversight of the GLAO to the GSG and the Coordinator for Graduate Student Life, who shall conduct periodic reviews of the office and be responsible for approving and overseeing the GLAO operating budget.
    \end{enumerate}
  \end{bylaws-number}
  \item Available Funds \hfill \\
  For the purpose of setting the GSG’s annual budget, the amount of GSAF funds available for the following fiscal year shall be determined by the calculation of anticipated fee revenue for that year. Should the actual fee revenue differ from the anticipated amount, the budget shall be adjusted according to the procedures set forth in 7.3.C.
  \item Restrictions on GSAF Expenditures \hfill \\
  The GSAF shall not be expended in a manner which discriminates on the basis of race, color, creed, sex, sexual orientation, marital status, personal appearance, age, national origin, political affiliation, physical or mental disability, or on the basis of the exercise of rights secured by the First Amendment of the United States Constitution, in accordance with the University of Maryland Human Relations Code.
  \item The authority to review and fund requests for specific graduate student events outside the context of regular GSG programming is delegated by the University Finance Committee to the GSG with the following stipulations:
  \begin{bylaws-number}
    \item The allocation process shall be conducted with the review and advice of the University Vice President for Student Affairs or his/her designee.
    \item The process of administering and allocating the GSAF should be consistent with the fee’s purposes and be fair and equitable.
    \item All graduate student groups registered in the Student Activities Reporting System (STARS), known as Graduate Student Organizations (GSO), and all students currently enrolled in a graduate degree granting program shall be given the opportunity to request funding for specific events from the GSG.
    \item The allocation process shall include a procedure whereby GSOs or students requesting funding will receive written explanations for any funding decisions, as well as procedures whereby decisions may be appealed.
    \item Records of the review and allocation process shall be kept by the GSG.
    \item Appropriate checks and balances shall be adopted to insure that funds are spent according to the specific purposes of the approved allocations, and that there is compliance with any restrictions on allocated funds.
    \item The funding process should be flexible enough to meet the changing needs of the graduate student body while providing for reasonable continuity in the funding of annual or recurring events traditionally supported by the GSAF.
  \end{bylaws-number}
\end{bylaws-number}

\section{Policy for the Recommendation of GSAF Changes}
The GSG and GLAO shall each receive a portion of the total GSAF. In order to request a change in the total amount of the GSAF, the GSG Assembly must approve the proposal following the procedures set forth below. :
\begin{bylaws-number}
  \item Procedures for Adjusting the GSG’s Portion of the GSAF
\begin{bylaws-number}
  \item The Vice President for Financial Affairs, in consultation with the President and the Executive Committee, shall be responsible for proposing any changes to the amount of the GSAF dedicated to the GSG. If an adjustment to the fee should be necessary, the Vice President for Financial Affairs shall forward the recommended change to the Rules Committee, who shall be responsible for creating a resolution calling for its approval.
  \item The Vice President for Legislative Affairs must notify all Representatives of the proposed fee adjustment no later than two weeks prior to the date when the vote on the recommendation is to be taken.
  \item No vote to recommend a change may be taken on a day in which there are no regularly scheduled Spring or Fall semester classes.
  \item In order for the fee adjustment to pass, it must be approved by a minimum two-thirds majority of all present and voting members of the Assembly.
  \item In the event the Assembly recommends the fee adjustment, the President shall forward the proposal to the Committee for the Review of Student Fees (CRSF).
\end{bylaws-number}
  \item Procedures for Adjusting the GLAO’s Portion of the GSAF
\begin{bylaws-number}
  \item Any recommendation to adjust the amount of the GSAF dedicated to the GLAO must be initiated by the GLAO or the GSG Executive Committee. If an adjustment to the fee should be necessary, the GLAO must submit to the GSG Executive Committee a proposal that includes, at minimum, the following information:
\begin{enumerate}[i]
  \item Background information on the GLAO,
  \item The GLAO’s current operating budget,
  \item The proposed adjustment to the fee,
  \item A new budget proposal detailing how the adjustment would affect the office’s spending,
  \item A rationale for the proposed adjustment,
  \item Potential outcome should the adjustment fail to be approved.
\end{enumerate}
  \item The proposal must be submitted to the Executive Committee no later than two months before the deadline for submitting proposed adjustments to the Committee for the Review of Student Fees (CRSF). The Executive Committee reserves the right to require more detailed or additional information from the GLAO as necessary.
  \item Within two weeks of its receipt, the Executive Committee shall vote whether or not to approve the adjustment proposal. If approved by the Executive Committee, the proposal shall be forwarded to the Rules Committee, which shall be responsible for creating a resolution calling for its approval.
  \item The Vice President for Legislative Affairs must notify all Representatives of the proposed fee adjustment no later than two weeks prior to the date when the vote on the recommendation is to be taken.
  \item No vote to recommend a change may be taken on a day in which there are no regularly scheduled Spring or Fall semester classes.
  \item In order for the fee adjustment to pass, it must be approved by a minimum two-thirds majority of all present and voting members of the Assembly.
  \item In the event the Assembly recommends the fee adjustment, the President shall forward the proposal to the CRSF.
\end{bylaws-number}
\end{bylaws-number}

\section{The Graduate Student Government Budget}
\begin{bylaws-number}
  \item The GSG Operating Budget
  \begin{bylaws-number}
    \item Each Spring semester, the Vice President for Financial Affairs, in consultation with the President and the Executive Committee, shall prepare an operating budget for the following fiscal year. After the conclusion of elections for the following year’s officers, the Vice President for Financial Affairs Elect is encouraged to participate in the crafting of the budget.
    \item The budget shall include:
    \begin{enumerate}[i]
      \item Salaries and stipends for elected and appointed officials of the GSG who are paid through the organization’s budget.
      \item Administrative and operating expenses for the GSG.
      \item Social events and other programming paid for by the GSG.
      \item Funds to support events organized by graduate students and funded through the Event Funding Request (EFR) process.
      \item Such other items and expenses as shall be deemed necessary to carrying out the mission of the GSG.
    \end{enumerate}
    \item The budget shall be submitted in line item format in accordance with the standard procedures and practices followed by University departments and units.
    \item The budget for each fiscal year shall be available to the Assembly for consideration no later than 1 June of the previous fiscal year.
  \end{bylaws-number}
  \item Procedures for Legislative Approval
  \begin{bylaws-number}
    \item The Vice President for Financial Affairs shall submit the completed budget to the Rules Committee. The committee shall report the budget to the Assembly, attached to an act calling for its adoption, which shall be sponsored by the Vice President for Financial Affairs. The act shall be immediately placed on the agenda of the next meeting of the Assembly.
    \item The Assembly shall have the right to amend the proposed budget by revising individual line items, though the total amount of the allocation must remain the same as originally proposed.
    \begin{enumerate}[i]
      \item The dollar amount of individual line items may be increased or decreased and lines may be deleted from or added to the budget.
      \item Amendments that affect a single line item change or a group of related or separate changes may be proposed, but in the latter case, any motion to divide the question and call for a single vote on each proposed change shall be automatically approved without being subject to a vote.
      \item All amendments shall be considered in accordance with the standing rules of the Assembly.
    \end{enumerate}
    \item Once all amendments have been resolved, the Assembly shall vote on whether to approve the budget as a whole. Approval of the budget requires a simple majority vote.
  \end{bylaws-number}
  \item Revision
  \begin{bylaws-number}
    \item At such times as it may be deemed necessary, the Vice President for Financial Affairs, in consultation with the President and the Executive Committee, may revise the budget to account for unplanned expenditures, shifting priorities, or unforeseen changes in the amount of money available to the GSG, subject to the provisions below.
    \item Changes to the budget shall constitute a revision requiring Assembly approval if such changes:
    \begin{enumerate}[i]
      \item Call for the introduction or deletion of any line items not in the primary budget as last approved or revised by the Assembly or for the transfer of funds from one category within the budget to another (except in such cases as dealt with in Article 7.3.C.5).
      \item Involve a change in the total amount of the operating budget as last approved or revised by the Assembly.
    \end{enumerate}
    \item Any such revision shall be subject to approval by a majority vote of the Assembly and shall follow the procedures outlined in Article 7.3.B.
    \item Changes which call for the transfer of funds between line items within the same category of the budget shall not constitute a revision and may be approved automatically by the Vice President for Financial Affairs, in consultation with the President and Executive Committee.
    \item If there are unspent funds remaining in any line item of the budget following the last regularly scheduled Assembly meeting of a session, the Vice President for Financial Affairs, in consultation with the President and Executive Committee, may move funds between budget lines or categories, subject to the following stipulations:
    \begin{enumerate}[i]
      \item Any transfers, along with their rationale, must be communicated to the Assembly via email, and must be summarized in the Vice President for Financial Affairs’ final report.
      \item Transfers may only be made into existing line items in the operating budget.
      \item Transfers may not be made to increase hourly wages or honoraria.
    \end{enumerate}
  \end{bylaws-number}
  \item Additional Restrictions on Expenditures
  \begin{bylaws-number}
    \item The budget shall not allocate funds for the purchase of bottled water less than one gallon, except as provided for under the conditions described below:
    \begin{enumerate}[i]
      \item The purchase is for an event funded through the EFR process; and
      \item The GSG or the planners of the event can demonstrate to the Budget and Finance Committee that there are no sustainable alternatives to bottled water, or that those alternatives present an undue financial or logistical burden; and
      \item The Budget and Finance Committee (BFC) approves of the exception.
    \end{enumerate}
  \end{bylaws-number}
 
  \item Spend Down
  \begin{bylaws-number}
    \item Spend Down” is defined as monies leftover in the GSG Budget at the end of Spring Semester before the end of the fiscal year (July 1), that cannot be utilized under GSG Spending Bylaws in any budgetary capacity.
    \item With the presentation of the Annual GSG Budget for the subsequent fiscal year in the May Assembly meeting, either the GSG President or Vice President of Financial Affair, will bring the projected spend down amount forward to the GSG Assembly for discussion for the allocation of its funds. Under circumstances where neither the GSG President nor the Vice President of Financial Affairs can present the projected spend down amount, the Chair of the Budget and Finance Committee may do so.
    \item The Assembly will generate a finalized list of potential allocations by the conclusion of the May General Assembly meeting, to be narrowed down by the executive committee. The options will be brought forth for a vote during the June General Assembly meeting where a majority vote will determine the allocation of the Spend Down funds.
    \item The decision to allocate the Spend Down funds to the majority-voted option may be debated upon the Assembly floor if the assembly determines that further discussion is warranted.
    \item Potential allocations of funds may be brought forth earlier than the May General Assembly Meeting to be considered for voting during June General Assembly Meeting.
    \item If in the event that it is to be predicted that quorum may not be reached in the June GSG General Assembly meeting to vote upon the Spend Down allocations, the assembly may vote during the May General Assembly Meeting to decide upon the allocation of the Spend Down monies.
    \item If quorum is not met in the June General Assembly meeting for which the voting of the Spend Down allocation is to occur, then the decision to allocate the Spend Down Funds will be decided upon by the Executive Committee.
  \end{bylaws-number}
\end{bylaws-number}

\section{Executive Compensation}
\begin{bylaws-number}
  \item Assistantships \hfill \\
The GSG shall provide the President and Director of Operations with full graduate assistantships to be awarded on a twelve month basis. The terms of the assistantships shall begin 1 July and run concurrently with the terms of office and the fiscal year.
\begin{bylaws-number}
  \item Stipend levels for these assistantships will be set at the rate for twelve month graduate assistantships as established by Stamp Student Union and Office of Campus Programs.
  \item The stipend associated with the President’s assistantship may only be withheld in conjunction with the resignation or removal from office by impeachment of the President.
\end{bylaws-number}
  \item The GSG will provide all Executives with honoraria payments, awarded on a semester basis, as part of the regular budgetary process. Honoraria shall be set at a minimum value of \$1,250 per semester. The Assembly may make more funds available for Executive honoraria at their discretion. Executive honoraria may only be withheld by a two-thirds majority vote of the Assembly or the resignation, or removal from office by impeachment, of the respective Executive.
\end{bylaws-number}

\section{Procedures for Funding Events}
\begin{bylaws-number}
  \item Proposals for graduate student event funding may be made by:
\begin{bylaws-number}
  \item Any graduate student organization (GSO) registered with the Office of Campus Programs in accordance with the Graduate Student Organizations registration guidelines. Eligible GSOs must have graduate students as their principal officers, and must have a minimum of fifty-percent of their student members be registered graduate students.
  \item Any student currently enrolled in a graduate degree granting program.
  \item Any University office or department, as long as the proposed event will benefit graduate students.
\end{bylaws-number}
  \item Funding shall not be allocated to events that meet the following criteria:
\begin{bylaws-number} 
  \item An event that denies participation or attendance to any currently enrolled graduate student. Participation and attendance must be open to all graduate students. When limits on the number of participants are necessary, the GSO or graduate student applying for funding (hereafter “planning entity”) must clearly state the maximum number of participants and the reason behind any restrictions, which must comply with the restrictions set forth in Article 7.1.C.
  \item An event that is engaged in activities that result in personal financial gain for individual members of the planning entity (not precluding the payment of wages or honoraria pursuant to a supervised contract for services).
  \item An event at which graduate students are not expected to constitute a majority of the attendees/participants, unless the event is intended to provide opportunities for developing student leadership and civic engagement skills or support student-run learning experiences outside of the formal classroom.
\end{bylaws-number}
  \item Funding Limits
\begin{bylaws-number}
  \item Each EFR will normally be limited to five-percent of the total amount of money allocated to the funding of EFRs by the GSG operating budget.
  \item A GSO or department may submit an EFR that requests more than five-percent of the EFR budget, though such requests must be formally presented to the Assembly for approval.
  \item No EFR submitted by an individual graduate student may be approved for more than five-percent by either the BFC or the Assembly.
  \item EFR events may only be funded from the budget of the fiscal year in which they occur.
  \item No BFC may approve funding for an event that takes place in a subsequent fiscal year.
\end{bylaws-number}
  \item Procedures for Considering Event Funding Requests
\begin{bylaws-number}
  \item In order to be eligible for funding, each planning entity requesting funds from the GSG must submit an EFR at least one month before the date of the event. Under extraordinary circumstances, the BFC may consider EFRs submitted after this deadline at its discretion. The GSG shall accept only one EFR per event. In the case that a revision is necessary, the BFC shall consider revised budgets provided the following:
\begin{enumerate}[i]
  \item The reason for and necessity of the change is fully explained in the revision.
  \item The revision is submitted sufficiently in advance of the event for the BFC to consider it.
\end{enumerate}
  \item Each EFR should include:
\begin{enumerate}[i]
  \item In the case of proposals by GSOs, the group’s vision, mission, and goals.
  \item In the case of proposals by individuals, a list of individuals involved in planning the event.
  \item A description of the event that includes its priorities or goals.
  \item A marketing plan.
  \item A listing of any fees that will be charged for attendance at the event.
  \item Whether or not the event is being co-sponsored by another organization. If so, the EFR should include the estimated or committed amount of that group’s contribution.
  \item The date, time, and location of the event.
  \item An assessment of who will most likely attend the event, the total number of attendees expected, and past attendance numbers, if applicable.
  \item A line item budget including a detailed description of each expenditure.
\end{enumerate}
  \item All EFRs must be submitted through the online EFR system.
  \item If the planning entity has other sources of funding, those amounts and any requirements for such funding must be disclosed, and contact numbers for any other funding entities must be included with the EFR. The BFC reserves the right to verify the accuracy of the information provided, and reevaluate funding decisions if new information becomes available. If the planning entity has access to significant funding from sources outside the GSG, the BFC or the Assembly should set GSG funding levels accordingly.
  \item Given the Assembly is in recess during the months of July and August, no EFRs for more than five-percent of the yearly EFR budget may be funded during that period. EFRs for events taking place in September that request more than five-percent may only be funded if there is adequate time for the Assembly to consider the EFR before the event takes places, according to the timelines established in Article 7.5.G.
\end{bylaws-number}
  \item Additional restrictions on EFR expenditures:
\begin{bylaws-number}
  \item Advertising \hfill \\
Any newspaper, poster, flyer, handbill, or other form of advertising for an event which has received funds from the GSG and is to be displayed on campus must have the following clause printed on it: “This event/program/function is funded in part by your Graduate Student Activities Fee and is open to the entire graduate student community.”
  \item Travel Expenses \hfill \\
Travel expenses, such as hotel accommodations and transportation costs, may be covered only in the context of an event with broad reaching impact and may not have as their primary purpose the financial or other material advantage of individual persons.
  \item The GSG will not fund the following:
\begin{enumerate}[i]
  \item Capital expenditures
  \item The purchase of alcoholic beverages, drugs, or other illegal substances
  \item Salaries
  \item Direct payments to philanthropic organizations or charities
  \item Clothing
  \item National and Regional Dues/Fees/Registrations
  \item Awards, Plaques, Certificates, or Trophies
  \item The purchase of bottled water less than one gallon, except as provided for under the conditions described in 7.3.D.1.
\end{enumerate}
\end{bylaws-number}
  \item Procedures for the Review of Funding Requests under Five-Percent
\begin{bylaws-number}
  \item The BFC will review all requests for funding that are under five-percent of the yearly EFR budget and decide what level of funding to award each request.
  \item All eligible GSOs, departments, or graduate students will be given the opportunity to present their funding requests in writing and, should they desire, in a formal presentation to the BFC. The planning entity may request such a presentation, or the BFC may ask the entity for a presentation (lasting no more than fifteen minutes) when the committee finds it necessary to facilitate the allocation decision process.
  \item The BFC will provide the planning entity with a decision within seven calendar days of the original submission of the EFR.
  \item Any urgent requests for monies in a time frame less than the required one month must be approved for consideration by the Vice President for Financial Affairs before coming before the BFC.
\end{bylaws-number}
  \item Procedures for the Review of Funding Requests over Five-Percent
\begin{bylaws-number}
  \item Providing that there are EFR funds still exceeding five-percent of the yearly EFR budget available, requests for funds exceeding five-percent of the yearly EFR budget must be presented to the Assembly for approval. Since the Assembly meets once each month during the session, requests for more than five-percent of the EFR budget must be submitted at least six weeks prior to the event date to allow for a presentation to be scheduled. The BFC may provisionally approve such an EFR for a lesser amount (not to exceed five-percent of the yearly EFR budget); should the planning entity decide before the date of its presentation to the Assembly to lower the request to the amount provisionally approved by the BFC, the entity may forego an appearance before the Assembly and be approved through the normal EFR process.
  \item The BFC shall submit to the Assembly its funding recommendations in the form of a report which shall be treated as a main motion on the agenda of the Assembly meeting in question.
  \item When the EFR comes before the Assembly for consideration, the Vice President for Financial Affairs or another member of the BFC shall present the committee report to the Assembly and explain its recommendations.
  \item After the presentation of the report, the planning entity must make its presentation no longer than five minutes and answer such questions as the Assembly may have.
  \item Upon the conclusion of the planning entity’s presentation, the Assembly shall immediately move to consideration of the question on whether to accept the committee’s recommendations. The Assembly shall debate the recommendations of the committee report according to its standard procedures, and amendments to them shall be in order.
  \item Once the time allotted for debate has elapsed and all amendments have been considered, the Assembly shall vote on the committee’s recommendations (as amended, if applicable).
  \item Once the total allotted amount has been decided upon, the planning entity will be immediately informed of the Assembly’s decision by the Presiding Officer of the Assembly.
  \item The BFC will have five days to specify which line items the money should be allocated toward, and to inform the planning entity of their decision.
\end{bylaws-number}
  \item EFR Appeal Process \hfill \\
When funds are denied for an event, the planning entity may appeal the decision. The process is as follows:
\begin{bylaws-number}
  \item Appeal process to the BFC
\begin{enumerate}[i]
  \item Any appeal must be filed within forty-eight hours of the announcement of the funding decision. The appeal must be submitted by email, and should clearly state the grounds for the appeal.
  \item The appeal should be carefully reasoned, detailing why the planning entity thinks that the decision is unfair or unfounded. Including possible changes to the event and EFR that might satisfy the BFC’s concerns is welcome and encouraged.
  \item The BFC will consider the appeal, make its decision, and notify the planning entity within forty-eight hours of its receipt of the appeal.
  \item In the event of a denial of the appeal, the organizers may appeal to the Assembly as described below.
\end{enumerate}
  \item Appeal process to the Assembly
\begin{enumerate}[i]
  \item Any appeal must be filed at least forty-eight hours prior to the Assembly meeting at which the appeal is to be presented. Because meetings are only held once a month during each session, planning entities should consider this when submitting their EFR.
  \item The appeal should be carefully reasoned, detailing why the planning entity thinks that the decision is unfair or unfounded.
  \item Once added to the Assembly meeting agenda, the planning entity will have four minutes to present its appeal and five minutes to answer the Assembly’s questions.
  \item The Assembly shall then review the appeal and vote on whether to uphold the BFC’s ruling. The Assembly’s decision shall be final, and not subject to appeal.
\end{enumerate}
\end{bylaws-number}
  \item Letter/Email of Award \hfill \\
When funds are approved for an event, the BFC will issue a letter/email of award to the planning entity. This letter of award will constitute the GSG’s official commitment to fund the event, and it should be accompanied by the approved budget when presented to other campus units for disbursement of funds.
  \item Post Event Report \hfill \\
After the completion of any event that has received EFR funding, the planning entity must submit a short description of the event and photographs, if available, to the GSG for posting on the GSG website within two weeks of the event. Failure to submit this report within four weeks of the event will result in automatic sanctions that make the planning entity ineligible for funding for a full calendar year.
  \item Sanctions
\begin{bylaws-number}
  \item If a planning entity does not submit a post event report within two weeks of its EFR event, the Vice President for Financial Affairs shall send an email to the planning entity reminding it of the required report and that failure to submit a report within four weeks of the event will result in sanctions. If a report is not submitted, the Vice President for Financial Affairs shall email the planning entity with the following information:
  \begin{enumerate}[i]
    \item An announcement of the start of the sanction period,
    \item The date on which the sanction will expire,
    \item An explanation of the appeal procedures outlined in Article 7.5.K.2.
  \end{enumerate}
    \item Appeals to these sanctions may be submitted to the Assembly for consideration at any time prior to the expiration of the period of ineligibility. Any such appeal shall be considered using the following procedures:
    \begin{enumerate}[i]
      \item The appeal must be filed with the Vice President for Legislative Affairs at least three weeks prior to the Assembly meeting at which the planning entity wishes to present its appeal. After verifying with the Vice President for Financial Affairs that the group has been sanctioned, the Vice President for Legislative Affairs shall schedule the appeal at an appropriate place in the agenda of the next Assembly meeting.
      \item At the meeting at which they are to present, the sanctioned planning entity will have four minutes to present its appeal and five minutes to answer the Assembly’s questions. The appeal should be carefully reasoned, detailing why the planning entity thinks that the decision is unfair or unfounded, or explaining why an exception to the GSG’s policies should be made. The Vice President for Financial Affairs, or his or her designee, should present the background for the BFC’s decision.
      \item The Assembly shall then review the appeal and vote on whether to lift the sanctions. The Assembly’s decision shall be final, and not subject to appeal.
    \end{enumerate}
  \end{bylaws-number}
  \item EFR Conflict of Interest Transparency \hfill \\
  When an Event Funding Request (EFR) is submitted the applicant Graduate Student Organization (GSO) must list all GSG members that are also members of their GSO. This is inclusive of program representatives and members the Budget and Finance Committee (BFC). GSG members that are members of the applicant GSO as defined by the organization’s bylaws and/or constitution, will not be allowed to vote on any funding decisions within the BFC (if they are committee members) or Assembly meetings. These GSG members must identify themselves before discussion begins at Assembly meetings and excuse themselves from closed discussion and voting at Assembly meetings for funding requests exceeding 5\% of the EFR budget, as detailed in 7.5.G.1. Failure of the GSO to identify GSG representative members may result in sanctions as detailed in
  \item BFC committee members that fail to identify themselves as concurrent members of the applicant GSO will be (1) removed from the BFC for the remainder of their tenure as a graduate student and (2) will not be allowed to vote on funding decisions for the remainder of their tenure as GSG representatives. GSG representatives that fail to identify themselves during Assembly meeting discussions detailed in 7.5.G.1 will not be allowed to vote on funding decisions for the remainder of their tenure as GSG representatives. GSG members who are also members of the applicant GSO as defined above, cannot represent the GSO for funding requests that require approval before the General Assembly or the BFC.
  \item BFC Best Practices \hfill \\
  The Budget and Finance Committee (BFC) will maintain a Best Practices document to be freely available to Graduate Student Organizations (GSOs) submitting Event Funding Requests (EFRs). These Best Practices will be guidelines for the operation of the BFC in reviewing and awarding EFRs and shall provide guidance to GSOs to aid the completion of successful applications. The Best Practices will be eligible for review and discussion within the BFC and should be regularly updated by the Vice President for Financial Affairs (VPFA) and the committee members according to the changing needs of the Graduate Student Government (GSG).
\end{bylaws-number}